\documentclass{article}

\usepackage[left=3cm,right=3cm,top=4cm,bottom=3cm]{geometry}
\usepackage{graphicx}
\usepackage{float}
\usepackage{algorithm}
\usepackage{algpseudocode}

\title{\textbf{The MECA Block Cipher}}
\author{John Holly}
\date{\today}

\thispagestyle{plain}
\graphicspath{ {../img/key-schedule}{../img/metamorphic-engine} }

\begin{document}

\maketitle

\algblock{Input}{EndInput}
\algnotext{EndInput}
\algblock{Output}{EndOutput}
\algnotext{EndOutput}
\newcommand{\Desc}[2]{\State \makebox[2em][l]{#1}#2}

\begin{abstract}
  \centering
  \begin{minipage}{\dimexpr\paperwidth-10cm}
    TODO
  \end{minipage}
\end{abstract}

\bigskip

\section{Introduction}

MECA is a novel block cipher utilizing second-order cellular automata (MECA) and metamorphic engines in both encryption/decryption rounds and key scheduling.

\subsection{Cellular automata}

Cellular automata are discrete systems that undergo state transitions by iterating over each cell within the given dimensionality by means of a local transition rule. The operating area for a local transition is the cell itself and a specified number of neighbors. This sliding window across the state can be unbalanced but generally has an equal number of neighbors on either side of the target cell. For the purpose of this paper, the cellular space is one-dimensional and the neighborhood is the cell with its left and right neighbors. All CAs within this paper are two-state CAs, that is each cell is either a 1 or a 0. When the neighborhood reaches the boundary of the cellular space, a boundary condition must be put in place. For the purpose of this paper, the boundary is periodic (cyclic); borrowing a neighbor from the opposite side of the cellular state. 

\subsubsection{First-order (CA)}

When talking about orders of CAs, this is simply referring to the number of state transitions (timesteps) considered during a transition. First-order CAs are only aware of the current timestep. 

A CA can be described as a quadruple $(\mathcal{L},\mathcal{S},\mathcal{N},f)$, where $\mathcal{L}$ is the cellular space (state size) in one dimension, $\mathcal{S}$ is the finite set of states, $\mathcal{N} = (\vec{v_1}, \vec{v_2}, ..., \vec{v_m})$ is the association of a single cell's neighborhood of $m$ cells (including the cell itself) within $\mathcal{L}$, $f: \mathcal{S}^{m}\rightarrow\mathcal{S}$ is the local transition rule of the CA.

Let $r$ be the rule number selected where $r$ is between 0 and 255, $\mathcal{P}$ is the set of all permutations for a given neighborhood size where $\mathcal{N} \ni \mathcal{P}$, $i$ is the index of $\mathcal{N}$ in $\mathcal{P}$.

$$
  \mathcal{S}^{m}\rightarrow\mathcal{S} = f(\mathcal{N}, r)
$$

The local transition is defined as the selected rule number bit-shifted right by the index of the $\mathcal{N}$ in $\mathcal{P}$ ANDed with 1.

$$
  f(\mathcal{N}, r) = r \gg i \mathbin{\&} 1
$$

\subsubsection{Second-order (MECA)}

A second-order CA is defined similarly, with the exception of it tracking two timesteps $\mathcal{S}_t$ and $\mathcal{S}_{t-1}$. The second-order local transition is an extension of the definition of the first-order $f$ above. If the cell in $\mathcal{S}_{t-1} \ne f$ then the cell is set to 1, otherwise 0. By swapping the end state and preceding state the evolution can be reversed.

\subsection{Metamorphic Engines (CLUs)}

TODO: Will there be three metamorphic engines? One for the binop of combining key with plaintext.

Metamorphism is a process that occurs in rocks where transformation to a rock of a different composition occurs. The data-dependent transformations in CA and MECA make this seem like a fitting nomenclature.

There are two metamorphic engines in use, one using irreversible first-order CA rules as a one-way hashing function for the key-generating-key (unscheduled, the one of $b$ bytes specified as a parameter), and one used during encryption/decryption rounds selecting the second-order rules per round. The first-order engine selects from 3 class 4 elementary CA rules. The second-order engine selects from 7 cherry-picked rules that also display globally chaotic behavior\cite{MECA-Properties}.

\section{Design and Motivation}

The design of MECA began with inspiration from a handful of other well known ciphers: RC6\cite{RC6} for its small size and simplicity, speed, and elegant extensibility to different word sizes, rounds, and key length, Stone Cipher-192 (SC-192)\cite{SC-192} for its crypto logic unit (metamorphic engine, or CLU), and a variety of papers I've read on novel ciphers utilizing first-order elementary cellular automata.

The MECA cipher was designed to be simplistic and easily understood to those without a deep knowledge of mathematics and cryptography from an implementation standpoint. The goals in mind were simplistic chaos that can be easily visualized, extensibility, efficiency in both hardware and software implementations, and exploitation of the reversibility of second-order cellular automata in place of traditional feistel networks.

\section{Details}

Similarly to RC5 and RC6, MECA is fully parameterized and can be specified as MECA-$w$/$r$/$b$ with the word size (registers) being $w$ bits, $r$ encryption rounds, and $b$ bytes in the encryption key. Metamorphic logic units and cellular automata (both first-order and second-order) are the primitive building blocks throughout the cipher. Much like feistel networks can be run in reverse with the same logic as forwards, MECAs can be run in reverse by simply swapping the pre-initial and initial states (the previous and current timesteps [state] of the cellular automata) the forward pass (encryption) converged to.

\subsubsection{First-order CLU}

The first-order CLU selects a class 4 elementary CA rule based on the remainder ($mod$ 3) of the key-generating-key word at index $i$ of a specific iteration in the scheduling algorithm XORed with the current key schedule key at index $j$. The mappings are shown in table \ref{tab:first-order-clu} and the rules are shown evolved with random initial conditions in figures \ref{fig:figure1}, \ref{fig:figure2}, \ref{fig:figure3}.

The metamorphic engine for CA rule selection is defined by the piecewise function below.

$$
  m(x) = \cases{ 54  & $x$ $mod$ $3 = 0$ \cr
                 110 & $x$ $mod$ $3 = 1$ \cr
                 137 & $x$ $mod$ $3 = 2$ }
$$

\begin{table}[h!]
  \begin{center}
    \caption{First-order CLU mapping}
    \label{tab:first-order-clu}
    \begin{tabular}{l|r} % <-- Alignments: 1st column left, 2nd middle and 3rd right, with vertical lines in between
      \textbf{Remainder} & \textbf{Rule}\\
      \hline
      0 & 54\\
      1 & 110\\
      2 & 137\\
    \end{tabular}
  \end{center}
\end{table}

\begin{figure}[H]
  \begin{center}
    \begin{minipage}{0.48\textwidth}
      \caption{Rule 54}
      \label{fig:figure1}
      \centering
      \includegraphics[scale=.5]{54.png}
    \end{minipage}
    \begin{minipage}{0.48\textwidth}
      \caption{Rule 110}
      \label{fig:figure2}
      \centering
      \includegraphics[scale=.5]{110.png}
    \end{minipage}
  \end{center}
\end{figure}
\begin{figure}[H]
  \begin{center}
    \begin{minipage}{0.48\textwidth}
      \caption{Rule 137}
      \label{fig:figure3}
      \centering
      \includegraphics[scale=.5]{137.png}
    \end{minipage}
  \end{center}
\end{figure}

\subsubsection{Second-order CLU}

The second-order CLU selects a chaotic second-order rule based on the remainder ($mod$ 7) of the current state of the MECA (timestep $t$). The rules are shown evolved with random initial conditions in figures \ref{fig:figure4}, \ref{fig:figure5}, \ref{fig:figure6}, \ref{fig:figure7}, \ref{fig:figure8}, \ref{fig:figure9}, \ref{fig:figure10}.

The metamorphic engine for MECA rule selection is defined by the piecewise function below.

$$
  m(x) = \cases{ 75  & $x$ $mod$ $7 = 0$ \cr
                 86  & $x$ $mod$ $7 = 1$ \cr
                 89  & $x$ $mod$ $7 = 2$ \cr
                 149 & $x$ $mod$ $7 = 3$ \cr
                 166 & $x$ $mod$ $7 = 4$ \cr
                 173 & $x$ $mod$ $7 = 5$ \cr
                 229 & $x$ $mod$ $7 = 6$ \cr }
$$


\begin{figure}[H]
  \begin{center}
    \begin{minipage}{0.48\textwidth}
      \caption{Rule 75}
      \label{fig:figure4}
      \centering
      \includegraphics[scale=.5]{75.png}
    \end{minipage}
    \begin{minipage}{0.48\textwidth}
      \caption{Rule 86}
      \label{fig:figure5}
      \centering
      \includegraphics[scale=.5]{86.png}
    \end{minipage}
  \end{center}
\end{figure}
\begin{figure}[H]
  \begin{center}
    \begin{minipage}{0.48\textwidth}
      \caption{Rule 89}
      \label{fig:figure6}
      \centering
      \includegraphics[scale=.5]{89.png}
    \end{minipage}
    \begin{minipage}{0.48\textwidth}
      \caption{Rule 149}
      \label{fig:figure7}
      \centering
      \includegraphics[scale=.5]{149.png}
    \end{minipage}
  \end{center}
\end{figure}
\begin{figure}[H]
  \begin{center}
    \begin{minipage}{0.48\textwidth}
      \caption{Rule 166}
      \label{fig:figure8}
      \centering
      \includegraphics[scale=.5]{166.png}
    \end{minipage}
    \begin{minipage}{0.48\textwidth}
      \caption{Rule 173}
      \label{fig:figure9}
      \centering
      \includegraphics[scale=.5]{173.png}
    \end{minipage}
  \end{center}
\end{figure}
\begin{figure}[H]
  \begin{center}
    \begin{minipage}{0.48\textwidth}
      \caption{Rule 229}
      \label{fig:figure10}
      \centering
      \includegraphics[scale=.5]{229.png}
    \end{minipage}
  \end{center}
\end{figure}

\subsection{Key Schedule}

The key schedule uses the same magic constants for initialization as RC6. Let $P$ be the binary expansion of $e-2$ and $Q$ is the binary expansion of $\varphi-1$, where $\varphi$ is the golden ratio.

The function for the outcome of a single CA timestep with a given rule is defined by the function below. Let $m(x)$ be the desired metamorphic rule and $c(S_i, m(x))$ is the execution of the elementary CA, where $S_i$ is the resulting round key (for a given iteration).

$$
 S_i = c(S_i, m(x)) 
$$

\begin{algorithm}[H]
  \begin{algorithmic}
    \caption{Key schedule for MECA-$w$/$r$/$b$}\label{alg:schedule}
    \Input
      \Desc{L}{$b$ byte key preloaded into $c$-word array}
      \Desc{r}{number of rounds}
    \EndInput
    \Output
      \Desc{$S$}{$w$-bit round keys $S_{0, ..., 2r - 1}$}
    \EndOutput
    \State $S_0 \gets P$
    \For{$i \gets 1$ to $2r-1$} 
      \State $S_i \gets S_{i-1} + Q$
    \EndFor
    \State $j \gets 0$
    \State $i \gets 0$
    \For{$s \gets 0$ to max(c, $2r$)}
      \State $S_i \gets c(S_i, m(L_j \oplus S_i))$ 
      \State $i \gets i+1$ $mod$ $2r-1$
      \State $j \gets j+1$ $mod$ $c-1$
    \EndFor
  \end{algorithmic}
\end{algorithm}

\subsection{Encryption}

The encryption processes takes in 4 $w$-bit words of plaintext and creates two $w$-bit MECAs ($A$ and $B$) by taking the first/last two words and using them as the initial/pre-initial states.

The function for the outcome of a single MECA timestep with a given rule is defined by the function below. Let $m(x)$ be the desired metamorphic rule and $z(X, m(x))$ is the execution of the MECA.

$$
	X = z(X, m(x)) 
$$

\begin{algorithm}[H]
  \begin{algorithmic}
    \caption{Encryption (forward evolution) for MECA-$w$/$r$/$b$}\label{alg:encryption}
    \Input
      \Desc{$P$}{4 $w$-bit words of plaintext}
      \Desc{$S$}{key schedule of 2$r$ $w$-bit words}
    \EndInput
    \Output
      \Desc{$C$}{4 $w$-bit words of ciphertext}
    \EndOutput
    \State $A_{t-1} \gets P_0$
    \State $A_{t} \gets P_1$
	\State $B_{t-1} \gets P_2$
	\State $B_{t} \gets P_3$
	\State $j \gets 0$
	\For{$i \gets 0$ to $r - 1$}
		\State $A_{t} \gets A_{t} \oplus L_j$
		\State $A \gets z(A, m(A))$
		\State $B_{t} \gets B_{t} \oplus L_{j+1}$
		\State $B \gets z(B, m(B))$
		\State $j \gets j + 2$
    \EndFor
    \State $C_0 \gets A_{t-1}$
    \State $C_1 \gets A_{t}$
    \State $C_2 \gets B_{t-1}$
    \State $C_3 \gets B_{t}$
  \end{algorithmic}
\end{algorithm}

\subsection{Decryption}

The decryption processes takes in 4 $w$-bit words of ciphertext and creates two $w$-bit MECAs ($A$ and $B$) by taking the first/last two words and using them as the initial/pre-initial states, they are reversed during decryption.

The function for the outcome of a single MECA timestep in reverse with a given rule is defined by the function below. Let $m(x)$ be the desired metamorphic rule and $z(X, m(x))$ is the execution of the MECA.

$$
	X = z(X, m(x)) 
$$

\begin{algorithm}[H]
  \begin{algorithmic}
    \caption{Decryption (reverse evolution) for MECA-$w$/$r$/$b$}\label{alg:decryption}
    \Input
      \Desc{$C$}{4 $w$-bit words of ciphertext}
      \Desc{$S$}{key schedule of 2$r$ $w$-bit words}
    \EndInput
    \Output
      \Desc{$P$}{4 $w$-bit words of plaintext}
    \EndOutput
    \State $A_{t-1} \gets C_1$
    \State $A_{t} \gets C_0$
	\State $B_{t-1} \gets C_3$
	\State $B_{t} \gets C_2$
	\State $j \gets 2r-1$
	\For{$i \gets 0$ to $r - 1$}
		\State $A \gets z(A, m(A))$
		\State $A_{t-1} \gets A_{t-1} \oplus L_j$
		\State $B \gets z(B, m(B))$
		\State $B_{t-1} \gets B_{t-1} \oplus L_{j-1}$
		\State $j \gets j - 2$
    \EndFor
    \State $P_0 \gets A_{t-1}$
    \State $P_1 \gets A_{t}$
    \State $P_2 \gets B_{t-1}$
    \State $P_3 \gets B_{t}$
  \end{algorithmic}
\end{algorithm}

\section{Performance}

At optimal throughput, here are is the runtime for encryption and decryption for various word sizes. These tests were conducted from a C++ implementation running on an Intel(R) Core(TM) i7-6700HQ CPU @ 2.60GHz under Linux kernel 5.13.0-27.

\begin{table}[h!]
  \begin{center}
    \caption{Performance of various word sizes}
    \label{tab:performance}
    \begin{tabular}{l|r}
      \textbf{Word size} & \textbf{Execution time}\\
      \hline
      16-bit & 23 us\\
      32-bit & 38 us\\
      64-bit & 65 us\\
    \end{tabular}
  \end{center}
\end{table}

\section{Implementation Considerations}

Implementing the MECA cipher can pose its own challenges in terms of performance. Firstly, when creating the second-order automata $A$ and $B$ during half-rounds, you will need to efficiently divide the block in half which (depending on word size) can be up to 128-bits each. Finding a suitable container for the state is key to performance and avoiding strings at all cost. My solution for the reference implementation was using C++'s STL bitset. This way, you may efficiently create two 128-bit wide bitsets and left-shift/OR the two 64-bit words efficiently into the state of $A$ and $B$.

TODO: Use array of function pointers for metamorphic engines and compare performance.

\section{Test Vectors}

\section{Security}

\section{Flexibility and Future Direction}

\section{Conclusions}

\section{Acknowledgements}

Michael Andrew Sanchez, my family, and those who put up with me.

\begin{thebibliography}{unsrt}
  \bibitem{class-4}
  Dhar, Avinash and Lakdawala, Porus and Mandal, Gautam and Wadia, Spenta, \emph{Universal Cellular Automata and Class 4}, 1994
  \bibitem{RC6}
  Ronald L. Rivest, M.J.B. Robshaw, R. Sidney, and Y.L. Yin, \emph{The RC6\texttrademark Block Cipher} (First Advanced Encryption Standard (AES) Conference, USA, 1998).
  \bibitem{SC-192}
  Magdy Saeb, \emph{The Stone Cipher-192 (SC-192): A Metamorphic Cipher} (IJCNS) International Journal of Computer and Network Security, 2009
  \bibitem{MECA-Properties}
  A. H. Encinas, A. M. d. Rey, J. L. P. Iglesias, G. R. Sánchez and A. Q. Dios, \emph{Cryptographic Properties of Second-Order Memory Elementary Cellular Automata} 2008 Third International Conference on Availability, Reliability and Security, 2008, pp. 741-745, doi: 10.1109/ARES.2008.114.
\end{thebibliography}
\end{document}